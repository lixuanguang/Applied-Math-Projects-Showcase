\documentclass[a4paper,8pt]{article}
\pagestyle{plain}


\usepackage[utf8]{inputenc}
\usepackage{fancyhdr}
\usepackage[margin=2cm,foot=1cm]{geometry}
\usepackage[T1]{fontenc}
\usepackage{hyperref}
\usepackage{ragged2e}
\usepackage{pgf-umlcd}
\usepackage{graphicx,multicol}

\hypersetup{
	colorlinks,
	linkcolor={blue!80!black},
	citecolor={green!80!black},
	urlcolor={red!80!black}
}


% Used to set numbering of the content
\setcounter{tocdepth}{2}
\setcounter{secnumdepth}{3}

%%
% Math packages
\usepackage[utf8]{inputenc}
\usepackage{mathtools}
\usepackage{amssymb}
\usepackage{bm}
\usepackage{amsthm}
\usepackage{bbm}
\usepackage{amsmath}
\usepackage{rotating}
\usepackage{mathrsfs}
\usepackage{tikz-cd}
\usepackage{float}
\usepackage{enumitem}

\graphicspath{ {./images/} }

%%
% Math Commands
\def\upint{\mathchoice%
	{\mkern13mu\overline{\vphantom{\intop}\mkern7mu}\mkern-20mu}%
	{\mkern7mu\overline{\vphantom{\intop}\mkern7mu}\mkern-14mu}%
	{\mkern7mu\overline{\vphantom{\intop}\mkern7mu}\mkern-14mu}%
	{\mkern7mu\overline{\vphantom{\intop}\mkern7mu}\mkern-14mu}%
	\int}
\def\lowint{\mkern3mu\underline{\vphantom{\intop}\mkern7mu}\mkern-10mu\int}
\usepackage{tikz}
\newcommand{\N}{\mathbb{N}}
\newcommand{\Lagr}{\mathcal{L}}
\newcommand{\R}{\mathbb{R}}
\newcommand{\Q}{\mathbb{Q}}
\newcommand{\C}{\mathbb{C}}
\newcommand{\x}{\mathbf{x}}
\newcommand{\F}{\mathbf{F}}
\newcommand{\f}{\mathbf{f}}
\newcommand{\y}{\mathbf{y}}
\renewcommand{\b}{\mathbf{b}}
\renewcommand{\c}{\mathbf{c}}
\renewcommand{\a}{\mathbf{a}}
\newcommand{\h}{\mathbf{h}}
\newcommand{\g}{\mathbf{g}}
\newcommand{\z}{\mathbf{z}}
\newcommand{\ze}{\mathbf{0}}
\newcommand{\Z}{\mathbb{Z}}
\newcommand{\norm}[1]{\left\lVert#1\right\rVert}
\newcommand{\abs}[1]{\left\lvert#1\right\rvert}
\newcommand{\brk}[1]{ \left[#1\right] }
\newcommand{\brc}[1]{ \left\{#1\right\} }
\newcommand{\paren}[1]{ \left(#1\right) }
\newcommand{\normop}[1]{\left\lVert#1\right\rVert_\text{op}}
\newcommand{\LL}{\mathcal{L}}
\newcommand{\uni}{\overset{\text{uni}}{\to}}
\DeclareMathOperator{\diam}{diam}
\newcommand{\Prr}[1]{\text{Pr}\left(#1\right)}
\newcommand{\code}[1]{\texttt{#1}}

\definecolor{dartmouthgreen}{rgb}{0.05, 0.5, 0.06}
\definecolor{egyptianblue}{rgb}{0.06, 0.2, 0.65}
\definecolor{dukeblue}{rgb}{0.0, 0.0, 0.61}
\definecolor{jazzberryjam}{rgb}{0.65, 0.04, 0.37}
\definecolor{magenta}{HTML}{EC008C}

%%
% Self-defined useful shortcuts
\newcommand{\isomorp}{\xrightarrow{\sim}}
\newcommand{\hlt}[1]{\textit{{\color{dukeblue}#1}}}
\newcommand{\impt}[1]{\textit{{\color{jazzberryjam}#1}}}
\newcommand{\qntype}[1]{\textit{{\color{dartmouthgreen}#1}}}
\newcommand{\soln}[1]{\textit{{\color{magenta}#1}}}
\newcommand{\gcds}[1]{\textnormal{gcd}#1}
\newcommand{\mins}[1]{\textnormal{min}#1}
\newcommand{\maxs}[1]{\textnormal{max}#1}
\newcommand{\lcms}[1]{\textnormal{lcm}#1}
\newcommand{\degs}[1]{\textnormal{deg}#1}
\newcommand{\exps}[1]{\textnormal{exp}#1}
\newcommand{\tors}[1]{\textnormal{Tor}#1}
\newcommand{\homs}[1]{\textnormal{Hom}#1}
\newcommand{\anns}[1]{\textnormal{Ann}#1}

%%
\setcounter{section}{0}
% Set up math tools
\theoremstyle{theorem}
\newtheorem{theorem}{Theorem}[section]
\newtheorem{corollary}[theorem]{Corollary}
\newtheorem{lemma}[theorem]{Lemma}
\newtheorem{proposition}[theorem]{Proposition}
\newtheorem{qnbank}[theorem]{Question Bank}
\let\oldqnbank\qnbank
\renewcommand{\qnbank}{\oldqnbank\normalfont}
\newtheorem{algorithm}[theorem]{Algorithm}
\let\oldalgorithm\algorithm
\renewcommand{\algorithm}{\oldalgorithm\normalfont}
\newtheorem{definition}[theorem]{Definition}
\let\olddefinition\definition
\renewcommand{\definition}{\olddefinition\normalfont}
\newtheorem{example}[theorem]{Example}
\let\oldexample\example
\renewcommand{\example}{\oldexample\normalfont}
\newtheorem{remark}[theorem]{Remark}
\let\oldremark\remark
\renewcommand{\remark}{\oldremark\normalfont}


\title{MA5251 Homework 2}
\author{Li Xuanguang, A0154735B}

\begin{document}
\maketitle

% Q1
\begin{enumerate}[label=Q\arabic	*.]
\setlength{\itemsep}{0pt}
\setcounter{enumi}{0}
\item
From lecture notes, we have the three-term recurrence relation
\begin{equation}
(n+1)L_{n+1}(x) = (2n+1)x L_n(x) - nL_{n-1}(x) \nonumber
\end{equation}
Differentiating once and by chain rule, we get
\begin{equation}
(n+1)L^{(1)}_{n+1}(x) = (2n+1) L_n(x) + (2n+1)x L^{(1)}_n(x) - nL^{(1)}_{n-1} (x) \nonumber
\end{equation}
Differentiating again, we get
\begin{align}
(n+1)L^{(2)}_{n+1}(x) &= (2n+1)L^{(1)}_n(x) + (2n+1)L^{(1)}_n(x) + (2n+1)x L^{(2)}_n(x) - nL^{(2)}_{n-1} (x) \nonumber \\
&= 2(2n+1)L^{(1)}_n(x) + (2n+1)x L^{(2)}_n(x) - nL^{(2)}_{n-1} (x) \nonumber
\end{align}
Hence, by the $m$-th differentiation, we get
\begin{equation}
(n+1)L^{(m)}_{n+1}(x) = m(2n+1)L^{(m-1)}_n(x) + (2n+1)x L^{(m)}_n(x) - nL^{(m)}_{n-1} (x) \nonumber
\end{equation}
From the second Proposition in Properties of Derivatives, Lecture 14 Slide 10, we can see that
\begin{equation}
L^{(1)}_{n+1}(x) - L^{(1)}_{n-1}(x) = (2n+1)L_{n}(x) \nonumber
\end{equation}
Differentiating $m-1$ times, this implies
\begin{equation}
L^{(m)}_{n+1}(x) - L^{(m)}_{n-1}(x) = (2n+1)L^{(m-1)}_{n}(x) \nonumber
\end{equation}
Substituting into the differentiated three-term recurrence relation, we get
\begin{align}
(n+1)L^{(m)}_{n+1}(x) = m[L^{(m)}_{n+1}(x) - L^{(m)}_{n-1}(x)] + (2n+1)x L^{(m)}_n(x) - nL^{(m)}_{n-1} (x)  \nonumber
\end{align}
Thus we get the three-term recurrence relation
\begin{align}
(n-m+1)L^{(m)}_{n+1}(x) &= (2n+1)x L^{(m)}_n(x) - (n+m)L^{(m)}_{n-1} (x)  \nonumber \\
L^{(m)}_{n+1}(x) &= \frac{2n+1}{n-m+1}x L^{(m)}_n(x) - \frac{n+m}{n-m+1}L^{(m)}_{n-1} (x) \nonumber
\end{align}
where $\alpha^{(m)}_n = \frac{2n+1}{n-m+1}$ and $\beta^{(m)}_n = \frac{n+m}{n-m+1}$ \\
For the initial condition, if $n < m$, then $L^{(m)}_n = 0$.\\
The first term is where $n=m$, then $L^{(m)}_m = (2m)!a_m = \frac{(2m)!}{2^m m!}$.\\
The next term is where $n = m+1$, then $L^{(m)}_{m+1} = a_{m+1}[(2m+2)!x - (2m+1)!(m+1)] = \frac{(2m+1)!}{2^m m!}x - \frac{(2m+1)!}{2^{m+1} m!}$.
\end{enumerate}

\newpage

% Q2
\begin{enumerate}[label=Q\arabic	*.]
\setlength{\itemsep}{0pt}
\setcounter{enumi}{1}
\item We prove both relations with induction.\\
From Lecture 14 Chebyshev Polynomials , we have the recurrence relation $T_{n+1}(x) = 2x T_n(x) - T_{n-1}(x)$.\\
Let $\theta = \arccos(x)$. Then $T_n(x) = \cos (n \theta)$, and this implies $\cos((n+1) \theta) = 2 \cos (\theta) \cos (n \theta) -  \cos ((n-1) \theta)$\\
\newline
Taking the derivative of $T_{n+1} (x)$ and $T_{n-1} (x)$,
\begin{align}
T^{(1)}_{n+1} (x) = - \theta^{(1)} (n+1) \sin ((n+1)\theta) \nonumber \\
T^{(1)}_{n-1} (x) = - \theta^{(1)} (n-1) \sin ((n-1)\theta) \nonumber
\end{align}
Hence we can prove the relation
\begin{align}
\frac{1}{n+1} T^{(1)}_{n+1} (x) - \frac{1}{n-1} T^{(1)}_{n-1} (x) &= - \theta^{(1)}[\sin((n+1)\theta) - \sin((n-1)\theta)] \nonumber \\
&= -\theta^{(1)} [2 \cos (n\theta)\sin(\theta)] \nonumber \\
&= \frac{\sin \theta}{\sqrt{1-(\cos\theta)^2}} 2 \cos (n \theta) \nonumber \\
&= 2 T_n(x) \nonumber
\end{align}
\newline
From Lecture 14 Chebyshev Polynomials, we have the inner product
\begin{equation}
(T_n(x), T_m(x)) = \int_{-1}^{1} T_m(x) T_n(x) \omega(x) \ dx = \frac{\pi}{2} c_n \delta_{mn} \nonumber
\end{equation}
where $c_n = 1 + \delta_{n0}$. Hence, we can derive the following results:
\begin{align}
(T_n(x), T_0(x)) &= \frac{\pi}{2} (1 + \delta_{n0}) \delta_{n0}, \ \ n \geq 1 \nonumber \\
(T_0(x), T_0(x)) &= \frac{\pi}{2} (1 + \delta_{00}) \delta_{00}, \ \ n \geq 1 \nonumber
\end{align}
\newline
For the first differential relation, we prove by induction.
\begin{align*}
T^{(1)}_2 = 2 \times 2 T_1 = 2 \cdot 2 \sum\limits^{1}_{\substack{k = 0 \\ k + n \ \text{odd}}} \frac{1}{c_k} T_k(x) \nonumber
\end{align*}
Assume $n = N \geq 2$. Then
\begin{align}
T^{(1)}_{N} (x) = 2N \sum\limits^{N-1}_{\substack{k = 0 \\ k + n \ \text{odd}}} \frac{1}{c_k} T_k(x) \nonumber
\end{align} 
For $n = N+1$, we use the result from the relation proved earlier.
\begin{align}
\frac{1}{N+1} T^{(1)}_{N+1} (x) &= 2 T_N(x) + \frac{1}{N-1} T^{(1)}_{N-1} (x) \nonumber \\
&= 2 T_N(x) + 2 \sum\limits^{N-1}_{\substack{k = 0 \\ k + N-1 \ \text{odd}}} \frac{1}{c_k} T_k(x) \nonumber \\
&= 2 T_N(x) + 2 \sum\limits^{N-1}_{\substack{k = 0 \\ k + N+1 \ \text{odd}}} \frac{1}{c_k} T_k(x) \ \ \text{\ \ , $k+N-1$ odd $\Rightarrow$ $k+N+1$ odd} \nonumber \\
&= 2 \sum\limits^{N}_{\substack{k = 0 \\ k + N+1 \ \text{odd}}} \frac{1}{c_k} T_k(x) \nonumber
\end{align}
Hence by induction,
\begin{equation}
T^{(1)}_{n} (x) = 2 n \sum\limits^{N-1}_{\substack{k = 0 \\ k + n \ \text{odd}}} \frac{1}{1 + \delta_{k0}} T_k(x) \nonumber
\end{equation}
\newpage
For the second differential relation, we prove by induction.
\begin{align*}
T^{(2)}_2 = 4 = \frac{1}{2} \cdot 2 \cdot (4-0) T_0 = \sum\limits^{0}_{\substack{k = 0 \\ k + n \ \text{even}}} \frac{n}{c_k} (n^2 - k^2) T_k(x) \nonumber
\end{align*}
Assume $n = N \geq 2$. Then
\begin{align}
T^{(2)}_{N} (x) = \sum\limits^{N-2}_{\substack{k = 0 \\ k + n \ \text{even}}} \frac{n}{c_k} (n^2 - k^2) T_k(x) \nonumber
\end{align}
For $n = N+1$, we use the result from the relation proved earlier.
\begin{align}
\frac{1}{N+1} T^{(2)}_{N+1} (x) &= \left(2 T_N(x) + \frac{1}{N-1} T^{(1)}_{N-1} (x)\right)^{(1)} \nonumber \\
&= 2 T^{(1)}_N(x) + \frac{1}{N-1} T^{(2)}_{N-1}(x) \nonumber \\
&= 4 N \sum\limits^{N-1}_{\substack{k = 0 \\ k + N \ \text{odd}}} \frac{1}{1 + \delta_{k0}} T_k(x) + \frac{1}{N-1} \sum\limits^{N-3}_{\substack{k = 0 \\ k + N - 1 \ \text{even}}} \frac{N-1}{c_k} ((N-1)^2 - k^2) T_k(x) \nonumber \\
&= \frac{1}{c_{N-1}} T_{N-1}(x) \cdot 4N + \sum\limits^{N-3}_{\substack{k = 0 \\ k + N - 3 \ \text{even}}} \frac{1}{c_k} ((N-1)^2 - k^2 + 4N) T_k(x) \nonumber \\
&= \frac{1}{c_{N-1}} T_{N-1}(x) \cdot [(N+1)^2 - (N-1)^2] + \sum\limits^{N-1}_{\substack{k = 0 \\ k + N + 1 \ \text{even}}} \frac{1}{c_k} ((N+1)^2 - k^2) T_k(x) \nonumber \\
&= \sum\limits^{N-1}_{\substack{k = 0 \\ k + N + 1 \ \text{even}}} \frac{1}{c_k} ((N+1)^2 - k^2) T_k(x) \nonumber
\end{align}
Hence by induction,
\begin{equation}
T^{(2)}_{n} (x) = \sum\limits^{n-2}_{\substack{k = 0 \\ k + n \ \text{even}}} \frac{1}{1 + \delta_{k0}} n(n^2 - k^2) T_k(x) \nonumber
\end{equation}
\end{enumerate}

\newpage
\begin{enumerate}[label=Q\arabic	*.]
\setlength{\itemsep}{0pt}
\setcounter{enumi}{2}
\item 
\begin{enumerate}[label=(\alph*)]
\setlength{\itemsep}{0pt}
\item Define the Sturm-Liouville operator
\begin{equation}
\Lagr f(x) = - \sqrt{1 - x^2} \frac{d}{dx} [\sqrt{1-x^2} f'(x)] \nonumber
\end{equation}
According to Sturm-Liouville equation, we have
\begin{equation}
\Lagr T_n(x) = n^2 T_n(x) \nonumber
\end{equation}
We also have
\begin{equation}
(\Lagr f, g) = (f, \Lagr g) \nonumber
\end{equation}
By completeness of basis $\{T_n(x)\}_{n \in \N}$ in $L^2_{\omega} (-1, 1)$ space, we can expand any function $f \in L^2_{\omega} (-1, 1)$:
\begin{equation}
f(x) = \sum\limits_{n = 0}^{+ \infty} \hat{f}_n T_n(x) \nonumber
\end{equation}
Note
\begin{align}
\norm{f - \pi_N f}^2_{0, \omega} &= \norm{\sum\limits_{n = N+1}^{+ \infty} \hat{f}_n T_n (x)}^2_{0,\omega} \nonumber \\
&= \sum\limits_{n = N+1}^{+ \infty} \frac{\pi}{2} \abs{\hat{f}_n}^2 \ \ \ \text{(as $\norm{T_n(x)}^2_{0, \omega} = \frac{\pi}{2}(1+\delta_{k0})$)}\nonumber \\
&= \sum\limits_{n = N+1}^{+ \infty} \frac{2}{\pi} \abs{\int_{-1}^1 f(x) T_n(x) \omega(x) \ dx}^2 \nonumber \\
&= \frac{2}{\pi} \sum\limits_{n = N+1}^{+ \infty} \abs{(f, T_n)_{\omega}}^2 \nonumber \\
&= \frac{2}{\pi} \sum\limits_{n = N+1}^{+ \infty} \abs{\frac{1}{n^2}(f, \Lagr T_n)_{\omega}}^2 \nonumber \\
&= \frac{2}{\pi} \sum\limits_{n = N+1}^{+ \infty} \abs{-\frac{1}{n^2}(\Lagr f, T_n)_{\omega}}^2 \nonumber \\
&= \frac{2}{\pi} \sum\limits_{n = N+1}^{+ \infty} \abs{\frac{1}{n^r}(\Lagr^{r/2} f, T_n)_{\omega}}^2 \ \ \ \text{(Assume $r$ is even)}\nonumber \\
&\leq \frac{2}{\pi} \frac{1}{N^{2r}} \sum\limits_{n = N+1}^{+ \infty}  \abs{(\Lagr^{r/2} f, T_n)_{\omega}}^2 \nonumber \\
&\leq \frac{1}{N^{2r}} \left[\frac{1}{\pi} \abs{(\Lagr^{r/2} f, T_n)_{\omega}}^2 + \sum\limits_{n=1}^{+ \infty} \frac{2}{\pi} \abs{(\Lagr^{r/2} f, T_n)_{\omega}}^2 \right] \nonumber \\
&\leq \frac{1}{N^{2r}} \norm{\Lagr^{r/2} f}^2_{0, \omega	} \nonumber \\
&\lesssim N^{-{2r}} \norm{f}^2_{r,\omega} \nonumber
\end{align}
Hence, $\norm{f - \pi_N f}_{0, \omega} \lesssim N^{-{r}} \norm{f}_{r,\omega}$.
\end{enumerate}
\end{enumerate}

\newpage
\begin{enumerate}[label=Q\arabic	*.]
\setlength{\itemsep}{0pt}
\setcounter{enumi}{2}
\item 
\begin{enumerate}[label=(\alph*)]
\setlength{\itemsep}{0pt}
\setcounter{enumii}{1}
\item $\forall p \in P_N$, the projection on Chebyshev Polynomial is $p(x) = \sum\limits_{n=0}^N \hat{p}_n T_n(x)$. \\
The derivative is then $p^{(1)}(x) = \sum\limits_{n=0}^N \hat{p}^{(1)}_n T^{(1)}_n(x)$, where $\hat{p}^{(1)}_n = \frac{2}{\pi(1+\delta_{0n})} \int^{1}_{-1} p^{(1)}(x) T_n(x) \omega(x) \ dx$.\\
The seminorm of $p(x)$ is defined as $\abs{p}^2_1 = \sum\limits_{n=0}^N \abs{\hat{p}^{(1)}_n}^2 \cdot \norm{T_n}^2_{L^2_{\omega}}$.\\
From the derivative $p^{(1)}(x)$ and differential relation of $T^{(1)}_n(x)$ from Q2, we can see that 
\begin{equation}
\hat{p}^{(1)}_n = \frac{2}{\pi(1+\delta_{0n})} \int^{1}_{-1} \left(\sum^N_{j=0} \hat{p}_j\right) \left(2j \sum\limits^{j-1}_{\substack{i=0 \\ i + j \ \text{odd}}} \frac{1}{1+\delta_{i0}}T_i(x) \right) T_n(x) \omega(x) \ dx \nonumber
\end{equation}
As $\{T_i(x)\}_{i \in \N}$ are orthogonal in $L^2_{\omega} (-1,1)$, if $i = n$, then the integral will be nonzero, thus $j \geq n + 1$ and $j + n$ is odd. Thus
\begin{align}
\hat{p}^{(1)}_n &= \frac{2}{\pi(1+\delta_{0n})} \left(\sum^N_{\substack{j = n+1 \\ j + n \ \text{odd}}} \hat{p}^{(1)}_j \cdot 2j \cdot \frac{1}{1+\delta_{n0}} \cdot \norm{T_n}^2_{L^2_{\omega}}\right) \nonumber \\
&= \frac{4}{\pi (1+\delta_{0n})^2} \left(\sum^N_{\substack{j = n+1 \\ j + n \ \text{odd}}} \hat{p}_j \cdot j \right) \cdot \norm{T_n}^2_{L^2_{\omega}} \nonumber
\end{align}
Thus we can see
\begin{align}
\abs{p}^2_1 &= \sum\limits_{n=0}^N \abs{\hat{p}^{(1)}_n}^2 \cdot \norm{T_n}^2_{L^2_{\omega}} \nonumber \\
&= \sum\limits_{n=0}^N \left(\frac{4}{\pi (1+\delta_{0n})^2} \right)^2 \left(\sum^N_{\substack{j = n+1 \\ j + n \ \text{odd}}} \hat{p}_j \cdot j \right)^2 \cdot \norm{T_n}^6_{L^2_{\omega}} \ \ \text{by earlier calculation} \nonumber \\
&= \sum\limits_{n=0}^N \left(\sum^N_{\substack{j = n+1 \\ j + n \ \text{odd}}} \hat{p}_j \cdot j \right)^2 \cdot \norm{T_n}^2_{L^2_{\omega}} \ \ \text{(as $\norm{T_n}^2_{L^2_{\omega}} = \frac{\pi}{2}(1+\delta_{0n})$)} \nonumber \\
&\leq \sum\limits_{n=0}^N \left(\sum^N_{j = 0} \hat{p}_j \cdot j \right)^2 \cdot \norm{T_n}^2_{L^2_{\omega}} \nonumber \\
&\leq \sum\limits_{n=0}^N \left(\sum^N_{j = 0} \hat{p}^2_j \right) \left(\sum^N_{j = 0} \cdot j \right) \cdot \norm{T_n}^2_{L^2_{\omega}} \ \ \text{by Cauchy-Schwarz} \nonumber \\
&\leq N^4 \sum\limits_{n=0}^N \left(\sum^N_{j = 0} \hat{p}^2_j \right) \cdot \norm{T_n}^2_{L^2_{\omega}} \ \ \text{by Cauchy-Schwarz} \nonumber \\
&\lesssim N^4 \norm{p}^2_{L^2_{\omega}} \nonumber
\end{align}
Hence, $\abs{p}_1 \lesssim N^2 \norm{p}_{L^2_{\omega}}$.\\
Note that for $n \in \N$, 
\begin{align}
\abs{p}_n &= \abs{p^{(1)}}_n-1 = \abs{p^{(2)}}_n-2 = \cdots = \abs{p^{(n-1)}}_1 \lesssim N^2 \norm{p^{n-1}}_{L^2_{\omega}}  \nonumber \\
\norm{p^{(n-1)}}_{L^2_{\omega}} &= \abs{p^{(n-2)}}_1 \lesssim N^2 \norm{p^{(n-2)}}_{L^2_{\omega}} \nonumber
\end{align}
Thus by these relations, we would have the error estimate
\begin{equation}
\norm{p}_{r, \omega} \lesssim N^{2r} \norm{p}_{L^2_{\omega}} = N^{2r} \norm{p}_{0,w\omega} \ \ \forall p \in P_N \nonumber
\end{equation}
\end{enumerate}
\end{enumerate}

\newpage
\begin{enumerate}[label=Q\arabic	*.]
\setlength{\itemsep}{0pt}
\setcounter{enumi}{2}
\item 
\begin{enumerate}[label=(\alph*)]
\setlength{\itemsep}{0pt}
\setcounter{enumii}{2}
\item From Q2, we have that $T^{(1)}_{n} (x) = 2 n \sum\limits^{N-1}_{\substack{k = 0 \\ k + n \ \text{odd}}} \frac{1}{1 + \delta_{k0}} T_k(x)$.\\
From part (a), we also have the expansion of $f$ as $f(x) = \sum\limits_{n = 0}^{+ \infty} \hat{f}_n T_n(x)$. Hence
\begin{align}
\partial_x f &= \sum\limits_{n=1}^{+ \infty} \hat{f}_n T^{(1)}_n(x) \nonumber \\	
&= \sum\limits_{n=1}^{+ \infty} \hat{f}_n \cdot 2 n \sum\limits^{n-1}_{\substack{k = 0 \\ k + n \ \text{odd}}} \frac{1}{1 + \delta_{k0}} T_k(x) \nonumber \\
&= \sum\limits_{k=0}^{+\infty} \frac{1}{1 + \delta_{k0}} \left(\sum\limits^{+\infty}_{\substack{n = k+1 \\ k + n \ \text{odd}}} 2n \cdot \hat{f}_n \right) T_k(x) \nonumber
\end{align}
Then 
\begin{equation}
\pi_N(\partial_x f) = \sum\limits_{k=0}^{N} \frac{1}{1 + \delta_{k0}} \left(\sum\limits^{+\infty}_{\substack{n = k+1 \\ k + n \ \text{odd}}} 2n \cdot \hat{f}_n \right) T_k(x) \nonumber
\end{equation}
\begin{align}
\partial_x(\pi_Nf) &= \sum\limits_{n=1}^N \hat{f}_n T^{(1)}_n (x) \nonumber \\
&= \sum\limits_{n=1}^N \hat{f}_n \cdot 2n \sum\limits^{n-1}_{\substack{k = 0 \\ k + n \ \text{odd}}} \frac{1}{1 + \delta_{k0}} T_k(x) \nonumber \\
&= \sum\limits^{N-1}_{k=0} \frac{1}{1 + \delta_{k0}} \left(\sum\limits^{N}_{\substack{n = k+1 \\ k + n \ \text{odd}}} 2n \cdot \hat{f}_n \right) T_k(x) \nonumber
\end{align}
Thus we can see that
\begin{align}
\pi_N(\partial_x f) - \partial_x(\pi_Nf) &= \sum\limits_{k=0}^{N} \frac{1}{1 + \delta_{k0}} \left(\sum\limits^{+\infty}_{\substack{n = k+1 \\ k + n \ \text{odd}}} 2n \cdot \hat{f}_n \right) T_k(x) - \sum\limits^{N-1}_{k=0} \frac{1}{1 + \delta_{k0}} \left(\sum\limits^{N}_{\substack{n = k+1 \\ k + n \ \text{odd}}} 2n \cdot \hat{f}_n \right) T_k(x) \nonumber \\
&= \sum\limits^{N}_{k=0} \frac{1}{1 + \delta_{k0}} \left(\sum\limits^{+ \infty}_{\substack{n = N+1 \\ k + n \ \text{odd}}} 2n \cdot \hat{f}_n \right) T_k(x) \nonumber
\end{align}
Note that
\begin{align}
\begin{cases}
\sum\limits^{+ \infty}_{\substack{n = N+1 \\ k + n \ \text{odd}}} 2n \cdot \hat{f}_n = \sum\limits^{+ \infty}_{\substack{n = N+1 \\ N + n \ \text{odd}}} 2n \cdot \hat{f}_n = \hat{f}^{(1)}_N \ \ \text{if $N+n$ is even} \\
\sum\limits^{+ \infty}_{\substack{n = N+1 \\ k + n \ \text{odd}}} 2n \cdot \hat{f}_n = \sum\limits^{+ \infty}_{\substack{n = N+2 \\ N + 1 + n \ \text{odd}}} 2n \cdot \hat{f}_n = \hat{f}^{(1)}_{N+1} \ \ \text{if $N+n$ is odd}
\end{cases} \nonumber
\end{align}
Thus,
\begin{align}
\norm{\pi_N(\partial_x f) - \partial_x(\pi_Nf)}^2_{L^2_{\omega}} &= \sum\limits_{k=0}^N \left(\frac{1}{1+\delta_{k0}} \sum\limits^{+\infty}_{\substack{n = k+1 \\ k + n \ \text{odd}}} 2n \cdot \hat{f}_n \right)^2 \cdot \frac{\pi}{2}(1+\delta_{k0}) \nonumber \\
&= \frac{\pi}{2} \sum\limits_{k=0}^N \frac{1}{1+\delta_{k0}} \left(\sum\limits^{+\infty}_{\substack{n = k+1 \\ k + n \ \text{odd}}} 2n \cdot \hat{f}_n \right)^2 \nonumber \\
&= 
\begin{cases}
\frac{\pi}{2} \sum\limits_{k=0}^N \frac{1}{1 + \delta_{k0}} \left(\hat{f}^{(1)}_N \right) ^2 \ \ \text{if $k+n$ is even} \\
\frac{\pi}{2} \sum\limits_{k=0}^N \frac{1}{1 + \delta_{k0}} \left(\hat{f}^{(1)}_{N+1} \right) ^2 \ \ \text{if $k+n$ is odd}
\end{cases} \nonumber
\end{align}
To estimate the sum, note that
\begin{align}
\abs{\hat{f}^{(1)}_N}^2 &= \left[ \left(\frac{\pi}{2} (1+\delta_{N0}) \right)^{-1} \int_{-1}^1 \partial_x f \cdot T_N(x) \ dx \right]^2 \nonumber \\
&= \frac{4}{\pi^2} \norm{(\partial_x f, T_N)}^2_{L^2_{\omega}} \nonumber \\
&= \frac{4}{\pi^2} \norm{-\frac{1}{N^2} (\partial_x f, \Lagr T_N)}^2_{L^2_{\omega}} \ \ \ \text{(By Sturm-Liouville Operator)} \nonumber \\
&\lesssim N^{-4} \norm{(\Lagr(\partial_x f), T_n)}^2_{L^2_{\omega}} \nonumber \\
&\lesssim N^{-4(r-1)/2} \norm{(\Lagr^{(r-1)/2}(\partial_x f), T_n)}^2_{L^2_{\omega}} \nonumber \\
&\lesssim N^{-4(r-1)/2} \norm{\Lagr^{(r-1)/2}(\partial_x f)}^2_{L^2_{\omega}} \norm{T_n}^2_{L^2_{\omega}} \ \ \ \text{(By Cauchy-Schwarz Inequality)} \nonumber \\
&\lesssim N^{-4(r-1)/2} \norm{\Lagr^{(r-1)/2}(\partial_x f)}^2_{L^2_{\omega}} \left(\frac{\pi}{2} \right)^2 \nonumber \\
&\lesssim N^{-2r+2} \norm{f}^2_{r, \omega} \nonumber
\end{align}
Similarly, 
\begin{equation}
\abs{\hat{f}^{(1)}_{N+1}}^2 \lesssim N^{-2r+2} \norm{f}^2_{r, \omega} \nonumber
\end{equation}
Thus,
\begin{align}
\norm{\pi_N(\partial_x f) - \partial_x(\pi_Nf)}^2_{L^2_{\omega}} &\lesssim \frac{\pi}{2} \left( \sum\limits_{k=0}^N \frac{1}{1+ \delta_{k0}} \right) N^{-2r+2} \norm{f}^2_{r, \omega} \nonumber \\
&\lesssim N^{3-2r} \norm{f}^2_{r, \omega} \nonumber
\end{align}
Therefore,
\begin{equation}
\norm{\pi_N(\partial_x f) - \partial_x(\pi_Nf)}_{L^2_{\omega}} \lesssim N^{3/2-r} \norm{f}_{r, \omega} \nonumber
\end{equation}
\end{enumerate}
\end{enumerate}

\newpage
\begin{enumerate}[label=Q\arabic	*.]
\setlength{\itemsep}{0pt}
\setcounter{enumi}{2}
\item 
\begin{enumerate}[label=(\alph*)]
\setlength{\itemsep}{0pt}
\setcounter{enumii}{3}
\item Note that
\begin{align}
\norm{f - \pi_N f}^2_{1, \omega} &= \norm{f - \pi_N f}^2_{L^2_{\omega}} + \abs{f - \pi_N f}^2_1 \nonumber \\
&= \norm{f - \pi_N f}^2_{L^2_{\omega}} + \norm{\partial_x f - \partial_x (\pi_N f)}^2_{L^2_{\omega}} \nonumber \\
&\leq \norm{f - \pi_N f}^2_{L^2_{\omega}} + \norm{\partial_x f - \pi_N (\partial_x f)}^2_{L^2_{\omega}} +  \norm{\pi_N (\partial_x f) - \partial_x (\pi_N f)}^2_{L^2_{\omega}} \nonumber \\
&\lesssim N^{-2r} \norm{f}^2_{r, \omega} + N^{-2(r-1)}\norm{\partial_x f}^2_{(r-1), \omega} + N^{3-2r} \norm{f}^2_{r, \omega} \nonumber \\
&\lesssim N^{3-2r} \norm{f}^2_{r, \omega} \nonumber
\end{align}
Hence,
\begin{equation}
\norm{f - \pi_N f}_{1, \omega} \lesssim N^{3/2-r} \norm{f}_{r, \omega} \nonumber
\end{equation}
\end{enumerate}
\end{enumerate}

\newpage
\begin{enumerate}[label=Q\arabic	*.]
\setlength{\itemsep}{0pt}
\setcounter{enumi}{3}
\item Note that
\begin{align}
p^{*}(-1) &= \int_{-1}^{-1} \pi_N u' - \frac{1}{2} p(1)\ dy = 0 \nonumber \\
p^{*}(1) &= \int_{1}^{-1} \pi_N u' - \frac{1}{2} p(1) \ dy \nonumber \\
&= \int_{-1}^{1} \pi_N u' \ dy = 0 \nonumber 
\end{align}
Hence, $p^{*}(-1) = p^{*}(1) = 0$.\\
Note that
\begin{align}
(u' - p^{*})' &= u' - \pi_N u' - \frac{1}{2} p(1) \nonumber \\
&= u' - \pi_N u' - \frac{1}{2} \int_{-1}^{1} \pi_N u' - u' \ dy \nonumber
\end{align}
As $u(-1) = u(1) = 0$, then $\int_{-1}^{1} u' \ dy = 0$. Thus,
\begin{align}
\abs{u-p^{*}}_{1, \omega} &\lesssim \norm{u' - \pi_N u'}_{0, \omega} \nonumber \\
&\lesssim N^{-r} \norm{u'}_{r, \omega} \nonumber
\end{align}
From Q3(a) result, where $\norm{f - \pi_N f}_{0, \omega} \lesssim N^{-r} \norm{f}_{r,\omega}$, $r$ is even, we can then show that
\begin{align}
\abs{u-p^{*}}_{1, \omega} &\lesssim N^{1-r} \norm{u}_{r, \omega} \nonumber
\end{align}
where $r$ is odd.
\end{enumerate}


\end{document}
